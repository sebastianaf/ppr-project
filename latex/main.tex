\documentclass{article}
\usepackage[utf8]{inputenc}
\usepackage[spanish]{babel}
\usepackage{amsfonts}
\usepackage{design_ASC}
\usepackage{caption}

\usepackage{amsmath}
\usepackage{graphicx}
\usepackage[colorinlistoftodos]{todonotes}

\usepackage{underscore}

\begin{document}
\begin{titlepage}
    \begin{center}
        \vspace*{1cm}

        \includegraphics[width=0.22\textwidth]{img/universidadDelValle.png}
        
        \vfill
        \textbf{Proyecto final - PPR}\\
        \vfill
        
        Estudiantes\\
        Jhon Henry Carabalí Miranda cod. 201910001\\
        Jose Manuel Madrid Torres cod. 201943827\\
        Sebastián Afanador Fontal cod. 1629587\\
        \vfill
        Profesor\\
        Profesor Juan Francisco Diaz Ph.D.\\
        
        \vfill

        \textbf{Programación por restricciones}
        
        \vfill
           
        Escuela de Ingeniería de Sistemas y Computación -EISC\\
        Facultad de Ingeniería\\
        Universidad del Valle\\
        Cali - Colombia\\
        \vfill
        \date{05 de Agosto del 2022}

    \end{center}
\end{titlepage}
\section{Modelo básico}
Un estudio de grabación desea optimizar el tiempo que contrata a un grupo de actores $A$ para grabar una telenovela titulada "Desenfreno de Pasiones", esta telenovela tiene un conjunto de escenas $E$ que deben ordenarse para cumplir con el requerimiento mencionado de manera que sea mínimo el tiempo que cada actor debe estar en el estudio. Dicho de otra manera el orden importa porque el tiempo de permanencia . \newline

A continuación se describen los elementos que componen el problema según el paradigma de programación por restricciones.

\subsection{Parámetros de entrada}
\begin{itemize}
    \item $ACTORES$: Es el conjunto de los nombres de los actores de la telenovela.
    \item $n\textunderscore actores \in \mathbb{N}$: Es la cantidad de actores de la telenovela $n(ACTORES)$, es decir, la cardinalidad de $ACTORES$.
    \item $Duracion$: Es el conjunto de las duraciones de las escenas de la telenovela en unidades de tiempo con $Duracion_i \in \mathbb{N}$ donde $i \in [1..n\textunderscore escenas]$.
    \item $n\textunderscore escenas \in \mathbb{N}$: Es la cantidad de escenas de la telenovela $n(Duracion)$, es decir, la cardinalidad de $Duracion$.
    \item $Escenas$: Es la participación que tienen los actores en las escenas con $escena_{i,j} \in \{0,1\}$ cuando $j < n\textunderscore escenas + 1 $ donde  $1$ representa que el actor participa en la escena y $0$ no participa. $escena_{i,j}$ representa la participación del $i-esimo$ actor con $i \in [1..n\textunderscore actores]$ para la $j-esima$ escena con $j \in [1..n\textunderscore escenas + 1]$ donde el valor $escena_{i,n\textunderscore escenas + 1}$ corresponde al precio que cobra el actor $i$ por unidad de tiempo.
\end{itemize}

\subsection{Variables}
\begin{itemize}
    \item $orden\textunderscore escenas$: Representa el orden que deben tener las escenas de la telenovela para para minimizar el $costo\textunderscore total$. Es el conjunto de las escenas donde $orden\textunderscore escenas_i \in [1..n\textunderscore escenas]$ con $i \in [1..n\textunderscore escenas]$ son cada una de las posiciones en orden cronológico de las escenas.
    \item $Escenas\textunderscore$: De manera similar que $Escenas$, $Escenas\textunderscore$ es la información de la participación de los actores en las escenas donde $escena\textunderscore_{i,j} \in \{0,1\}$ representa la participación del $i-esimo$ actor con $i \in [1..n\textunderscore actores]$ para la $j-esima$ escena con $j \in [1..n\textunderscore escenas + 1]$ donde  $1$ representa que el actor participa en la escena y $0$ no participa. $Escenas\textunderscore$ se ordenará con respecto a $orden\textunderscore escenas$. Esta variable se crea con el fin de mostrar la matriz solución y así dar una mejor representación de la solución de un problema.
    \item $costo\textunderscore x\textunderscore actor$: Son los valores que cobrarán cada uno de los actores con $costo\textunderscore x\textunderscore actor_i \in \mathbb{N}$ donde $ i \in [1..n\textunderscore actores]$ con base al $orden\textunderscore escenas$ que minimice el $costo\textunderscore total$.
    \item $costo\textunderscore total \in \mathbb{N}$: Esta variable representa el costo total mínimo que debería tener la grabación de la telenovela.
    \item $costo\textunderscore x\textunderscore escena$: Son los valores que costarán cada una de las escenas con $costo\textunderscore x\textunderscore escena_i > 0$ donde $i \in [1..n\textunderscore escenas]$ con base al $orden\textunderscore escenas$ que minimice el $costo\textunderscore total$.
    \item $costo\textunderscore maximo\textunderscore x\textunderscore escena$: Es el conjunto de los valores que en teoría costarían cada una de las escenas si todos los actores tuvieran participación, $costo\textunderscore maximo\textunderscore x\textunderscore escena_i \in \mathbb{N}$ con $i \in [1..n\textunderscore escenas]$ es el costo máximo teórico de la escena $i$, esta variable se crea con el fin de mostrar el cálculo de la variable $costo\textunderscore maximo$.
    \item $costo\textunderscore maximo \in \mathbb{N}$: Esta variable representa el costo máximo que suman el total de los valores de $costo\textunderscore maximo\textunderscore x\textunderscore escena$.
\end{itemize}

\subsection{Funciones}
\begin{itemize}
    \item $primera\textunderscore escena(actor)$: Es la primera escena en la que participa un $actor \in [1..n\textunderscore actores]$ con base al nuevo orden de las escenas dado por $Escenas\textunderscore$.
    \item $ultima\textunderscore escena(actor)$: Es la primera escena en la que participa un $actor \in [1..n\textunderscore actores]$ con base al nuevo orden de las escenas dado por $Escenas\textunderscore$.
\end{itemize}

\subsection{Restricciones}
\begin{itemize}
    \item $\forall i \nexists j$ tal que $i = j$ \newline donde $i \land j \in orden\textunderscore escenas$
    \item $\forall i 1 \leq orden\textunderscore escenas_i \leq n\textunderscore escenas$  donde $i \in orden\textunderscore escenas$.
    \item $\forall i \forall j$  $Escenas\textunderscore_{i,j} = Escenas[i,orden\textunderscore escenas_{j}]$ \newline donde $i \in [1..n\textunderscore actores] \land j \in [1..n\textunderscore escenas]$.
    \item $\forall i\sum_{j=primera\textunderscore escena(i)}^{ultima\textunderscore escena(i)}$  $Duracion[orden\textunderscore escenas_j] * Escenas_{i,n\textunderscore escenas+1} = costo\textunderscore x\textunderscore actor$\newline
          donde $i \in [1..n\textunderscore actores]$.
    \item $\forall i [\sum_{j=1}^{n\textunderscore actores}$  $Escenas\textunderscore[j,orden\textunderscore escenas_i] * Escenas{j,n\textunderscore escenas+1}]*Duracion[orden\textunderscore escenas_i] = costo\textunderscore x\textunderscore escena$\newline
          donde $i \in [1..n\textunderscore escenas]$.
    \item $\forall i [\sum_{j=1}^{n\textunderscore actores}$  $Escenas[j,n\textunderscore escenas+1]] * Duracion_i = costo\textunderscore maximo\textunderscore x\textunderscore escena$\newline
          donde $i \in [1..n\textunderscore escenas]$.
\end{itemize}

\subsection{Función objetivo}
\begin{itemize}
    \item $min(\sum_{i=1}^{n\textunderscore actores}costo\textunderscore x\textunderscore actor$)
\end{itemize}

\section{Modelo extendido}
A continuación se describe el contenido del modelo extendido. \newline

\subsection{Parámetros de entrada}
A las entradas del modelo básico se le adicionan los siguientes parámetros.
\begin{itemize}
    \item $Disponibilidad$: Es la información de las restricciones de los actores especificados en el conjunto de $ACTORES$. $Disponibilidad_{i,j}$ con $i \in [1..n\textunderscore actores]$ y con $j \in [1..2]$ representa la cantidad de horas máxima que actor $i$ de nombre $Disponibilidad_{i,1} \in ACTORES$ puede estár disponible para la grabación de la telenovela con tiempo $Disponibilidad_{i,2} \in \mathbb{N} $. El caso en que la disponibilidad del actor $i$  $Disponibilidad_{i,2} = 0$ significa que no tiene restricciones de tiempo.
    \item $Evitar:$ Es la información de las parejas de actores que no 
    pueden estar (cualquiera que sea la razón) en la misma escena donde $Evitar_{i,j}$ con $i \in \mathbb{N}$ y con $j \in [1..2]$ representa la información de una pareja con $Evitar_{i,1} \in ACTORES$ los primeros actores y $Evitar_{i,2} \in ACTORES$ los segundos actores.
\end{itemize}

\subsection{Variables}
A las variables del modelo básico se le agregan las siguientes:
\begin{itemize}
    \item $tiempo\textunderscore x\textunderscore actor$: Son los tiempos que debe permanecer cada actor en las escenas para grabar donde $tiempo\textunderscore  x\textunderscore actor_i \in \mathbb{N}$ con $i \in [1..n\textunderscore actores]$ es el tiempo con respecto al orden inicial de las escenas.
    \item $tiempo\textunderscore min\textunderscore x\textunderscore actor$: Son los tiempos minimos que debe permanecer cada actor en las escenas para grabar donde $tiempo\textunderscore min\textunderscore x\textunderscore actor_i \in \mathbb{N}$ con $i \in [1..n\textunderscore actores]$ es el tiempo con respecto al $orden\textunderscore escenas$ que minimice el $costo\textunderscore total$.
    \item 
\end{itemize}

\subsection{Restricciones}
A las restricciones del modelo básico se le agregan las siguientes:
\begin{itemize}
    \item $\forall i\sum_{j=1}^{n\textunderscore escenas}$  $Duracion_j * Escenas_{i,j} = tiempo\textunderscore x\textunderscore actor$\newline
          donde $i \in [1..n\textunderscore actores]$.
    \item $\forall i\sum_{j=1}^{n\textunderscore escenas}$  $Duracion[orden\textunderscore escenas_j] * Escenas\textunderscore_{i,j} = tiempo\textunderscore min\textunderscore x\textunderscore actor$\newline
          donde $i \in [1..n\textunderscore actores]$.
\end{itemize}

\end{document}