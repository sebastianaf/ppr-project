\documentclass{article}
\usepackage[utf8]{inputenc}
\usepackage[spanish]{babel}
\usepackage{amsfonts}
\usepackage{design_ASC}
\usepackage{caption}

\usepackage{amsmath}
\usepackage{graphicx}
\usepackage[colorinlistoftodos]{todonotes}

\usepackage{underscore}

\begin{document}
\begin{titlepage}
    \begin{center}
        \vspace*{1cm}

        \includegraphics[width=0.22\textwidth]{img/universidadDelValle.png}
        
        \vfill
        \textbf{Proyecto final - PPR}\\
        \vfill
        
        Estudiantes\\
        Jhon Henry Carabalí Miranda cod. 201910001\\
        Jose Manuel Madrid Torres cod. 201943827\\
        Sebastián Afanador Fontal cod. 1629587\\
        \vfill
        Profesor\\
        Profesor Juan Francisco Diaz Ph.D.\\
        
        \vfill

        \textbf{Programación por restricciones}
        
        \vfill
           
        Escuela de Ingeniería de Sistemas y Computación -EISC\\
        Facultad de Ingeniería\\
        Universidad del Valle\\
        Cali - Colombia\\
        \vfill
        \date{05 de Agosto del 2022}

    \end{center}
\end{titlepage}
\section{Primera parte del Proyecto}
\subsection{Modelo}
Un estudio de grabación desea optimizar el tiempo que contrata a un grupo de actores $A$ para grabar una telenovela titulada "Desenfreno de Pasiones", esta telenovela tiene un conjunto de escenas $E$ que deben ordenarse para cumplir con el requerimiento mencionado de manera que sea mínimo el tiempo que cada actor debe estar en el estudio. Dicho de otra manera el orden importa porque el tiempo de permanencia . \newline

A continuación se describen los elementos que componen el problema según el paradigma de programación por restricciones.

\subsubsection{Parámetros}
\begin{itemize}
    \item $ACTORES$: Los nombres de los actores de la telenovela.
    \item $n\textunderscore actores$: Cantidad de actores de la telenovela $n(ACTORES)$ referenciando la cardinalidad de $ACTORES$.
    \item $Duracion$: Las duraciones de las escenas de la telenovela en unidades de tiempo.
    \item $n\textunderscore escenas$: La cantidad de escenas de la telenovela $n(Duracion)$ referenciando la cardinalidad de $Duracion$.
    \item $Escenas$: Con $escena_{ij} \in \{0,1\}$ la participación del $i-esimo$ actor con $i \in [1..n\textunderscore actores]$ para la $j-esima$ escena con $j \in [1..n\textunderscore escenas]$ donde $1$ representa que el actor participa en la escena y $0$ no participa.    
\end{itemize}

\subsubsection{Variables}
\begin{itemize}
    \item $Orden$: Es una lista de las escenas donde $orden_i \in [1..n\textunderscore escenas]$ con $i \in [1..n\textunderscore escenas]$ son cada una de las posiciones en orden cronológico de las escenas.
    \item $CostoTotal = [\sum_{i=1}^{n\textunderscore actores} costo_i * \sum_{j=1}^{n\textunderscore escenas}duraciones_j * participacion_{ij} ]   > 0$. \newline\newline
          Es el costo total de grabar la telenovela definido como las sumas de los productos del costo de cada actor con la suma de las duraciones de sus ensayos/grabaciones.
\end{itemize}

\subsubsection{Restricciones}


\subsubsection{Función objetivo}

\subsubsection{Extras}
Listar el máximo ahorro posible a partir de la diferencia entre el costo total máximo y el costo total mínimo.

\end{document}