%Documento de Latex para Taller 2 Programación con restricciones
\documentclass{article}
\usepackage[utf8]{inputenc}
\usepackage[spanish]{babel}
\usepackage{amsfonts}
\usepackage{design_ASC}
\usepackage{caption}

\usepackage{amsmath}
\usepackage{graphicx}
\usepackage[colorinlistoftodos]{todonotes}



\begin{document}
\begin{titlepage}
    \begin{center}
        \vspace*{1cm}

        \includegraphics[width=0.22\textwidth]{img/universidadDelValle.png}
        
        \vfill
        \textbf{Proyecto final - PPR}\\
        \vfill
        
        Estudiantes\\
        Jhon Henry Carabalí Miranda cod. 201910001\\
        Jose Manuel Madrid Torres cod. 201943827\\
        Sebastián Afanador Fontal cod. 1629587\\
        \vfill
        Profesor\\
        Profesor Juan Francisco Diaz Ph.D.\\
        
        \vfill

        \textbf{Programación por restricciones}
        
        \vfill
           
        Escuela de Ingeniería de Sistemas y Computación -EISC\\
        Facultad de Ingeniería\\
        Universidad del Valle\\
        Cali - Colombia\\
        \vfill
        \date{05 de Agosto del 2022}

    \end{center}
\end{titlepage}
\subsection{Modelo 1}
Este modelo genérico propone el uso de ingredientes dinámicos para la producción de cualquier compuesto producido a partir de la mezcla de varios ingredientes teniendo en cuenta sus aportes con respecto a una distribución de restricciones.\\

\subsubsection{Parámetros}
Sea $p$ el total de galones que se desea producir de un compuesto. Sean $A$ los ingredientes del compuesto que se desea preparar donde $a_i \in A$ con $i \in [1..n]$ es un ingrediente. Sea $AB_{ab}$ el aporte $b$ con $b \in [1,m]$ de un ingrediente $a$ con $a \in [1,n]$ para el compuesto que se desea preparar. Sean $C$ los precios donde $c_i \in C$ con $i \in [1,n]$ es el costo por galón del ingrediente $i$.\\

Sean $D$ los límites mínimos y máximos de aporte total donde $d_i \in D$ con $i \in [1,m]$ son los límites inferiones y superiores $\{g,l\}$ del aporte $i$.\\

\subsubsection{Variables}
Sean $E$ las cantidades necesarias de los ingredientes donde $e_i \in E$ con $i \in [1,n]$ es la cantidad de un ingrediente para producir el compuesto teniendo en cuenta sus aportes.\\

\subsubsection{Restricciones}
\begin{enumerate}
    \item $\sum_{i=1}^n c_i= p$
\end{enumerate}

\subsubsection{Función objetivo}
\begin{enumerate}
    \item $min(\sum_{i=1}^n c_i*e_i)$
\end{enumerate}

\end{document}