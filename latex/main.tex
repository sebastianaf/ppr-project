\documentclass{article}
\usepackage[utf8]{inputenc}
\usepackage[spanish]{babel}
\usepackage{amsfonts}
\usepackage{design_ASC}
\usepackage{caption}

\usepackage{amsmath}
\usepackage{graphicx}
\usepackage[colorinlistoftodos]{todonotes}

\usepackage{underscore}

\begin{document}
\begin{titlepage}
    \begin{center}
        \vspace*{1cm}

        \includegraphics[width=0.22\textwidth]{img/universidadDelValle.png}
        
        \vfill
        \textbf{Proyecto final - PPR}\\
        \vfill
        
        Estudiantes\\
        Jhon Henry Carabalí Miranda cod. 201910001\\
        Jose Manuel Madrid Torres cod. 201943827\\
        Sebastián Afanador Fontal cod. 1629587\\
        \vfill
        Profesor\\
        Profesor Juan Francisco Diaz Ph.D.\\
        
        \vfill

        \textbf{Programación por restricciones}
        
        \vfill
           
        Escuela de Ingeniería de Sistemas y Computación -EISC\\
        Facultad de Ingeniería\\
        Universidad del Valle\\
        Cali - Colombia\\
        \vfill
        \date{05 de Agosto del 2022}

    \end{center}
\end{titlepage}
\section{Modelo básico}
Un estudio de grabación desea optimizar el tiempo que contrata a un grupo de actores $A$ para grabar una telenovela titulada "Desenfreno de Pasiones", esta telenovela tiene un conjunto de escenas $E$ que deben ordenarse para cumplir con el requerimiento mencionado de manera que sea mínimo el tiempo que cada actor debe estar en el estudio. Dicho de otra manera el orden importa porque el tiempo de permanencia . \newline

A continuación se describen los elementos que componen el problema según el paradigma de programación por restricciones.

\subsection{Parámetros}
\begin{itemize}
    \item $ACTORES$: Son los nombres de los actores de la telenovela.
    \item $n\textunderscore actores$: Es la cantidad de actores de la telenovela $n(ACTORES)$, es decir, la cardinalidad de $ACTORES$.
    \item $Duracion$: Son las duraciones de las escenas de la telenovela en unidades de tiempo.
    \item $n\textunderscore escenas$: Son la cantidad de escenas de la telenovela $n(Duracion)$, es decir, la cardinalidad de $Duracion$.
    \item $Escenas$: Con $escena_{i,j} \in \{0,1\}$ cuando $j < n\textunderscore escenas + 1 $ donde  $1$ representa que el actor participa en la escena y $0$ no participa. $escena_{i,j}$ representa la participación del $i-esimo$ actor con $i \in [1..n\textunderscore actores]$ para la $j-esima$ escena con $j \in [1..n\textunderscore escenas + 1]$ donde el valor $escena_{i,n\textunderscore escenas + 1}$ corresponde al precio que cobra el actor $i$ por unidad de tiempo.
\end{itemize}

\subsection{Variables}
\begin{itemize}
    \item $costo\textunderscore total$: Esta variable representa el costo mínimo que debería tener la grabación de la telenovela.
    \item $orden\textunderscore escenas$: Representa el orden que deben tener las escenas de la telenovela para para minimizar el $costo\textunderscore total$. Es una lista de las escenas donde $orden\textunderscore escenas_i \in [1..n\textunderscore escenas]$ con $i \in [1..n\textunderscore escenas]$ son cada una de las posiciones en orden cronológico de las escenas.
    \item $Escenas\textunderscore$: De manera similar que $Escenas$, $Escenas\textunderscore$ es la información de la participación de los actores en las escenas ordenada con respecto a $orden\textunderscore escenas$. Esta variable se crea con el fin de mostrar una mejor representación de la solución de un problema.
    \item $costo\textunderscore x\textunderscore actor$: Este es el valor que cobrará cada $actor \in [1..n\textunderscore actores]$ con base al $orden\textunderscore escenas$ que minimice el $costo\textunderscore total$.
\end{itemize}

\subsection{Funciones}
\begin{itemize}
    \item $primera\textunderscore escena(actor)$: Es la primera escena en la que participa un $actor \in [1..n\textunderscore actores]$ con base al nuevo orden de las escenas dado por $Escenas\textunderscore$.
    \item $ultima\textunderscore escena(actor)$: Es la primera escena en la que participa un $actor \in [1..n\textunderscore actores]$ con base al nuevo orden de las escenas dado por $Escenas\textunderscore$.
\end{itemize}

\subsection{Restricciones}
\begin{itemize}
    \item $costo\textunderscore x\textunderscore actor = \forall i\sum_{j=1}^{n\textunderscore escenas} Duracion[orden\textunderscore escenas_j] * Escenas_{i,n\textunderscore escenas+1}$\newline
          donde $primera\textunderscore escena(i) \leq i \leq ultima\textunderscore escena(i)$
\end{itemize}

\subsection{Función objetivo}

\subsection{Extras}
Listar el máximo ahorro posible a partir de la diferencia entre el costo total máximo y el costo total mínimo.

\section{Modelo extendido}
A continuación se describe el contenido del modelo extendido. \newline


\end{document}